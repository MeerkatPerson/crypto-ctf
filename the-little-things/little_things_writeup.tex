\documentclass{article}
\usepackage[utf8]{inputenc}
\usepackage{amsmath}
\usepackage{hyperref}
\title{Breaking a Few Trivial RSA Padding Schemes}
\setlength\parindent{0pt}
\begin{document}
\maketitle

\section{Introduction}

This challenge is actually composed of five small challenges: the flag is broken into five chunks, and each chunk is encrypted using a different padding scheme. The given public key's public exponent $e$ is equal to 3, meaning that as soon as we have figured out a way of breaking each of the padding schemes, we can simply take the third root to obtain the respective chunk of the flag.

\medskip

I cracked chunks 1 to 4 in crack.py, for the fifth part I applied Coppersmith's theorem and relied on SageMath, see part5.sage.

\section{Chunk 1}

Here, no padding at all has been applied. We can simply take the third root and exit.

\section{Chunk 2}

With this chunk/encryption method, the padding consists in multiplying the chunk with a fixed number r. Now we can take advantage of the fact that RSA is multiplicative. Thus, computing $c*(r^{-1})^e$ mod $n$, which is equal to $(m*r)^e*(r^{-1})^e$ mod $n$, actually gives us $m^e$ mod $n$, and we can again take the third root to retrieve the chunk.

\section{Chunk 3}

Next, we have a padding strategy that fills with zeroes. Adding additional zeroes however reduces to the same case as with chunk 2 (multiplication with a fixed number), since it corresponds to multiplying by a power of 2. The power of two we can derive since the flag is broken into equal-length chunks. Thus, set $r=2^{l}$ and  proceed as in case 2.

\section{Chunk 4}

The fourth chunk's padding consists in repeating the chunk itself a number of times: $m'=m||m||...|m$. Mathematically, that is equivalent to $m'=m*10..010..01..01$, where each block of the form $0..1$ is of length $\lceil log_2(m) \rceil$. Therefore, this case again reduces to case 2.

\section{Chunk 5}

Here, I applied Coppersmith's theorem for finding a polynomial root, i.e., I cheated?? Might add more info about Coppersmith's here in the future. I derived my approach from \href{https://nars1st.tumblr.com/post/179527220982/whistle-crypto-250-220}{this blog post}.

\end{document}