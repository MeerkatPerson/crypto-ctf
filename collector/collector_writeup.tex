\documentclass{article}
\usepackage[utf8]{inputenc}
\usepackage{amsmath}
\usepackage{hyperref}
\title{Performing the simplest version of Hast{\aa}d's Broadcast Attack, and reversing OAEP}
\setlength\parindent{0pt}
\begin{document}
\maketitle

\section{Introduction}

This challenge is doubly educational: we have to apply the simplest version of Hast{\aa}d's broadcast attack AND reverse padding. Firstly though, some suitable cipher-key-pairs had to be harvested from the server (i.e., I queried the server until I got three key-cipher-pairs where the public encryption exponent was equal to 3). Apparently I was lazy and did this by hand. In the follow-up challenge (noble collector), I automated this querying procedure.

\section{H{\aa}stad's Broadcast - special case with identical messages}

In this version of the H{\aa}stad Broadcast attack, the same message is encrypted multiple times using the same public exponent $e$. In the standard case, $e=3$ and the same message has been sent thrice, e.g., the same invite has been sent in identical form (without personal addressing - how impolite!) to three different people. While the public exponent used for RSA encryption was the same for all three of them, the respective moduli $n_1$, $n_2$, $n_3$ were different.\medskip

The attack is possible thanks to the following \emph{Lemma}: ``If a message is encrypted with the same exponent $e$ but $e$ different moduli $n_i$, we can recover the message.'' - Here's the proof given in the lecture slides:\medskip

Firstly, among the three moduli there might be a pair that is not co-prime. Then we are done immediately: we have found a prime factor. Thus, in the following we will without loss of generality assume that $n_1$, $n_2$, $n_3$ are pairwise co-prime and we have a system of congruences of the form $c_i \equiv m_i^{e}$ mod $n_i$, $i \in \{1,...,e\}$.\medskip

Now, put $x=m^{e}$ and solve via CRT. This gives us a unique solution $m^{e}$ mod $\prod n_i$. Now we observe: $m$ must be smaller than each of the $n_i$. From this follows that $m^{e} < \prod n_i \implies m^{e}$ mod $\prod n_i = m^{e}$. Hence, we can just compute the third root, like we did in previous challenges and retrieve $m$. 

\section{OAEP - Optimal Asymmetric Encryption Padding}

To make the challenge, well, a bit more challenging we now also have to reverse the padding that was applied to $m$ before encryption. This padding is OAEP, the current (gold?) standard of padding when it comes to asymmetric encryption. Reversing it requires either building from the code provided with the challenge or following the instructions in \href{https://en.wikipedia.org/wiki/Optimal_asymmetric_encryption_padding}{the Wikipedia article}. Also this was the first of the challenges I completed as part of this course and thus I was still commenting my code really neatly, sigh!

\end{document}