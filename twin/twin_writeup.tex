\documentclass{article}
\usepackage[utf8]{inputenc}
\usepackage{amsmath}
\usepackage{hyperref}
\title{The Common Modulus Attack on RSA}
\setlength\parindent{0pt}
\begin{document}
\maketitle

\section{Introduction}

In this challenge, we deploy Bezout's identity to read a message sent to \emph{two} different people ($m_1 = m_2 = m$), i.e., encrypted under two different RSA keys that share the same modulus ($n_1 = n_2$). 

\medskip

\emph{Thm. (Bezout)}: Let $a, b$ integers with greatest common divisor $d$. Then there exist integers $x, y$ s.t. $ax + by = d$.  

\section{The Solution}

Firstly, we deploy the Extended Euclidean Algorithm (EEA, see \texttt{crt.py}) to find $d$ as well as the integers $x, y$ from Bezout's identity. $a, b$ in this case are the two RSA keys' public exponents $e_1, e_2$.

\medskip

Next, observe the obvious fact that $c_i = m^{e_i}$, for $i = 1, 2$. 
Taken together, we can compute (assuming $s < 0$ and $t > 0$):

\begin{align*}
    (c_{1}^{-1})^{|s|}\cdot c_2^{t} &\equiv ((m^{e_1})^{-1})^{|s|}(m^{e_2})^{t}\\
                                    &\equiv m^{se_1 + te_2} \\
                                    &\equiv m^d \medspace mod \medspace n 
\end{align*}

Now, a tiny twist comes in. In the most basic form of a common modulus attack, the two RSA keys' public exponents $e_1, e_2$ are co-prime, which means that $m^d = m^1 = m$. In our case however, $d = 17$. Thus, we aren't done yet! Luckily though, $m^{17}$ is apparently smaller than $n$, since taking the 17-th root of $m^{17} \medspace mod \medspace n$ gives us the flag.

\end{document}