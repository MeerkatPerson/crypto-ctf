\documentclass{article}
\usepackage[utf8]{inputenc}
\usepackage{amsmath}
\usepackage{amsfonts} % for "\mathbb" macro
\usepackage{amssymb}
\usepackage{hyperref}
\title{Wiener Attack + Factoring $n$ given $d$}
\setlength\parindent{0pt}
\begin{document}
\maketitle

\section{Introduction}

Honestly looking at this again months after completing the challenge I'm not sure I solved it in the intended way - did it simply require a common modulus attack (as in the ``twin'' challenge)? Anyways, if so I really took the scenic route here, mounting Wiener's attack AND factoring $n$ given the secret exponent $d$.\\

What we have given (``chall.py'', ``ciphers'', ``key1.pem'', ``key2.pem'') is a message, split in two, and each of the two parts encrypted using one of the keys. The two keys share the same modulus $n$, but have different values for $e$. So regarding my remark above, unsure if a common modulus attack would have been possible since the two keys were used to encrypt different messages rather than the same. I used Wiener's attack to break ``key1.pem'' (see the additional document ``wiener.tex''/``wiener.pdf'' for details), and then factored $n$ using the private exponent $d_1$.

\section{Theorem of Secret Parameters: \\ Break RSA using $d$}

The \emph{Theorem of Secret Parameters} states that given one entry of the private key ($p$,$q$,$\phi(n)$,$d$), and the public key, we can efficiently compute the full private key.\\

We know that by construction, $e\cdot d - 1 = x\cdot \phi(n)$ ($d$ is the multiplicative inverse of $e$ modulo $\phi(n)$).\\

Begin with the observation that since $p$, $q$ are odd, $\phi(n)=(p-1)(q-1)$ is a product of two even numbers and thus itself even. We can thus write 

\begin{align*}
    e \cdot d-1 &= x \cdot \phi(n) \\
                &= 2^s \cdot k \\
\end{align*}

I.e., we have decomposed $e \cdot d-1$ into $s$, its so-called multiplicity of two (the power of two in its prime factorization) and $k$, its odd part.\\

Now we will repeat the following procedure until we have found a factor of $n$:

\begin{enumerate}
    \item Pick a random value $0 < a < n$.
    \item Check if $gcd(a,n) > 1$. Since $n=pq$, the only greatest common divisor any number can share with $n$ is $\in \{1,p,q,n\}$. We have picked $a < n$, so if $gcd(a,n) > 1$, it has to be $\in \{p,q\}$ and we are done.
    \item Otherwise, check for each $i = 0, ..., s-1$ if $gcd((a^k)^{2^i}-1, n) \ne \{1,n\}$. If so, we have found a factor of $n$ and can exit.
\end{enumerate}

Amazingly, the success probability for each iteration of this procedure is $\frac{1}{2}$! How come? We begin with the following:

\begin{align*}
    gcd(a,n) = 1 &\implies (a^k)^{2^s} = a^{x\cdot \phi(n)} = (a^{\phi(n)})^x 
                 \equiv   1 \mod n \\
                 &\implies (a^k)^{2^s} - 1 \equiv 0 \mod n \\
                 &\implies gcd((a^k)^{2^s}-1, n) = n
\end{align*}

We have here used Fermat's Little Theorem. This congruence also holds modulo $p$ and $q$, $n$s co-prime factors (by the Chinese Remainder Theorem, I think). Now, enter group theory. In RSA, we operate in $\mathbb{Z}_{n}^{*}$, where the asterisk means we are only interested in elements that have a multiplicative inverse. By CRT, $\mathbb{Z}_{n}^{*} \cong \mathbb{Z}_{p}^{*} \times \mathbb{Z}_{q}^{*}$. \\

$\mathbb{Z}_{p}^{*}$ and $\mathbb{Z}_{q}^{*}$ are subgroups of $\mathbb{Z}_{n}^{*}$. We know that the order of $a^k$, $o(a^k)$, defined as $(a^k)^{o(a^k)} \equiv 1$, is equal to $2^s$ in $\mathbb{Z}_{n}^{*}$. In $\mathbb{Z}_{p}^{*}$, $\mathbb{Z}_{q}^{*}$, the order of $a^k$ may be smaller (or for sure is?). Luckily, by Lagrange's theorem, the order of $a^k$ in either of the two subgroups has to divide its order in $\mathbb{Z}_{n}^{*}$. \\

Let $g$, $h$ be generators of $\mathbb{Z}_{p}^{*}$, $\mathbb{Z}_{q}^{*}$ respectively. Then $a^k \cong (g^y, h^z)$ in $\mathbb{Z}_{p}^{*} \times \mathbb{Z}_{q}^{*}$. Let $o(g^y)=2^{l_1}$ and $o(h^z)=2^{l2}$. If $l_1 \neq l_2$, we have a success. Assume without loss of generality $l_1 < l_2$, then

\begin{align*}
    (a^k)^{2^{l_1}} \equiv 1 \mod p \\
    (a^k)^{2^{l_2}} \equiv j \mod q, j \neq 1 \\
    p | (a^k)^{2^{l_1}} - 1 \\
    q \nmid p | (a^k)^{2^{l_1}} - 1 \\
    \implies gcd((a^k)^{2^{l_1}} - 1, n) = p 
\end{align*}

\end{document}